\documentclass{article}
\usepackage[utf8]{inputenc}
\usepackage{listings}
\usepackage{color}


\definecolor{dkgreen}{rgb}{0,0.6,0}
\definecolor{gray}{rgb}{0.5,0.5,0.5}
\definecolor{mauve}{rgb}{0.58,0,0.82}

\lstset{frame=tb,
  language=C,
  aboveskip=3mm,
  belowskip=3mm,
  showstringspaces=false,
  columns=flexible,
  basicstyle={\small\ttfamily},
  numbers=none,
  numberstyle=\tiny\color{gray},
  keywordstyle=\color{blue},
  commentstyle=\color{dkgreen},
  stringstyle=\color{mauve},
  breaklines=true,
  breakatwhitespace=true,
  tabsize=3
}

\title{Practical 2}
\author{Nguyen Tien Cong}
\date{May 2022}


\begin{document}

\maketitle

\section{Introduction}
In this practical work, we write a C program that show the IP address of the domain name.
\\*
- Use the function gethostbyname() to resolve a domain name.
\\* - Show the resolved IP address

\section{Implementation}
\textbf{2.1 Include library}
\\*
This is very important step, without library, the program won't able to be executed!
\\*
Here is all the library that we need to include in out program: 
\begin{lstlisting}
#include <stdio.h>
#include <stdlib.h>
#include <unistd.h>
#include <errno.h>
#include <netdb.h>
#include <sys/types.h>
#include <sys/socket.h>
#include <netinet/in.h>
#include <arpa/inet.h>
#include <string.h>
\end{lstlisting}
\\*
\textbf{2.2 Get input name from user}
\begin{lstlisting}
   char hostname[256];
   scanf("%256s", hostname);
   printf("Host name : %s \n",hostname);
\end{lstlisting}
\\*
The array "char hostname[256]" is to store the input of the user
\\*
In "scanf" we need to limit the size of the input (256 bytes) to avoid buffer overflow attack.
\\*
\begin{lstlisting}
   host_name = gethostbyname(hostname);
    if (host_name == NULL){
        printf("Invalid host name!\n");
        exit(1);
    }
\end{lstlisting}
\\*

This step is to check the host name that user input is valid or not. If it not valid, it will send a message and close the program
\\*
\textbf{2.3 Resolved the Ip address (IPv4)}
\begin{lstlisting}
   else{
        for(int i=0; host_name->h_addr_list[i] != NULL; i++){
            printf("IP address: %s \n", inet_ntoa(*(struct in_addr *) (host_name->h_addr_list[i])));
        }
    }
\end{lstlisting}
This code is to print all the IP address of a domain name

\section{Result}
\textbf{3.1 On laptop}
\\*
First, we complile the file:
\begin{verbatim}
    gcc 02.practical.work.gethostbyname.c
\end{verbatim}
Then run it with the input: "google.com"
\begin{verbatim}
    ./a.out
\end{verbatim}
And the result:
\begin{verbatim}
    Enter host domain name: google.com
    Host name : google.com 
    IP address: 172.217.25.14 
\end{verbatim}

\textbf{3.2 On VPS}
\\*
First, we need to connect to the server
\begin{verbatim}
    root@{server ip address}
\end{verbatim}
\\*
Then we compile the C file and run it like we did above
\\*
With the input :"google.com"
\begin{verbatim}
    Enter host domain name: google.com
    Host name : google.com 
    IP address: 142.251.10.139 
    IP address: 142.251.10.138 
    IP address: 142.251.10.113 
    IP address: 142.251.10.102 
    IP address: 142.251.10.100 
    IP address: 142.251.10.101
\end{verbatim}
\\*
\textbf{3.3 Explain}
\\*
As you can see, the result on my local laptop and on my VPS is different. Because the geological distance can make a domain have a multiple addresses.

\end{document}
\documentclass{article}
\usepackage[utf8]{inputenc}
\usepackage{listings}
\usepackage{color}


\definecolor{dkgreen}{rgb}{0,0.6,0}
\definecolor{gray}{rgb}{0.5,0.5,0.5}
\definecolor{mauve}{rgb}{0.58,0,0.82}

\lstset{frame=tb,
  language=C,
  aboveskip=3mm,
  belowskip=3mm,
  showstringspaces=false,
  columns=flexible,
  basicstyle={\small\ttfamily},
  numbers=none,
  numberstyle=\tiny\color{gray},
  keywordstyle=\color{blue},
  commentstyle=\color{dkgreen},
  stringstyle=\color{mauve},
  breaklines=true,
  breakatwhitespace=true,
  tabsize=3
}

\title{Practical 2}
\author{Nguyen Tien Cong}
\date{May 2022}


\begin{document}

\maketitle

\section{Introduction}
In this practical work, we write a C program that show the IP address of the domain name.
\\*
- Use the function gethostbyname() to resolve a domain name.
\\* - Show the resolved IP address

\section{Implementation}
\textbf{2.1 Include library}
\\*
This is very important step, without library, the program won't able to be executed!
\\*
Here is all the library that we need to include in out program: 
\begin{lstlisting}
#include <stdio.h>
#include <stdlib.h>
#include <unistd.h>
#include <errno.h>
#include <netdb.h>
#include <sys/types.h>
#include <sys/socket.h>
#include <netinet/in.h>
#include <arpa/inet.h>
#include <string.h>
\end{lstlisting}
\\*
\textbf{2.2 Get input name from user}
\begin{lstlisting}
   char hostname[256];
   scanf("%256s", hostname);
   printf("Host name : %s \n",hostname);
\end{lstlisting}
\\*
The array "char hostname[256]" is to store the input of the user
\\*
In "scanf" we need to limit the size of the input (256 bytes) to avoid buffer overflow attack.
\\*
\begin{lstlisting}
   host_name = gethostbyname(hostname);
    if (host_name == NULL){
        printf("Invalid host name!\n");
        exit(1);
    }
\end{lstlisting}
\\*

This step is to check the host name that user input is valid or not. If it not valid, it will send a message and close the program
\\*
\textbf{2.3 Resolved the Ip address (IPv4)}
\begin{lstlisting}
   else{
        for(int i=0; host_name->h_addr_list[i] != NULL; i++){
            printf("IP address: %s \n", inet_ntoa(*(struct in_addr *) (host_name->h_addr_list[i])));
        }
    }
\end{lstlisting}
This code is to print all the IP address of a domain name

\section{Result}
\textbf{3.1 On laptop}
\\*
First, we complile the file:
\begin{verbatim}
    gcc 02.practical.work.gethostbyname.c
\end{verbatim}
Then run it with the input: "google.com"
\begin{verbatim}
    ./a.out
\end{verbatim}
And the result:
\begin{verbatim}
    Enter host domain name: google.com
    Host name : google.com 
    IP address: 172.217.25.14 
\end{verbatim}

\textbf{3.2 On VPS}
\\*
First, we need to connect to the server
\begin{verbatim}
    root@{server ip address}
\end{verbatim}
\\*
Then we compile the C file and run it like we did above
\\*
With the input :"google.com"\documentclass{article}
\usepackage[utf8]{inputenc}
\usepackage{listings}
\usepackage{color}


\definecolor{dkgreen}{rgb}{0,0.6,0}
\definecolor{gray}{rgb}{0.5,0.5,0.5}
\definecolor{mauve}{rgb}{0.58,0,0.82}

\lstset{frame=tb,
  language=C,
  aboveskip=3mm,
  belowskip=3mm,
  showstringspaces=false,
  columns=flexible,
  basicstyle={\small\ttfamily},
  numbers=none,
  numberstyle=\tiny\color{gray},
  keywordstyle=\color{blue},
  commentstyle=\color{dkgreen},
  stringstyle=\color{mauve},
  breaklines=true,
  breakatwhitespace=true,
  tabsize=3
}

\title{Practical 2}
\author{Nguyen Tien Cong}
\date{May 2022}


\begin{document}

\maketitle

\section{Introduction}
In this practical work, we write a C program that show the IP address of the domain name.
\\*
- Use the function gethostbyname() to resolve a domain name.
\\* - Show the resolved IP address

\section{Implementation}
\textbf{2.1 Include library}
\\*
This is very important step, without library, the program won't able to be executed!
\\*
Here is all the library that we need to include in out program: 
\begin{lstlisting}
#include <stdio.h>
#include <stdlib.h>
#include <unistd.h>
#include <errno.h>
#include <netdb.h>
#include <sys/types.h>
#include <sys/socket.h>
#include <netinet/in.h>
#include <arpa/inet.h>
#include <string.h>
\end{lstlisting}
\\*
\textbf{2.2 Get input name from user}
\begin{lstlisting}
   char hostname[256];
   scanf("%256s", hostname);
   printf("Host name : %s \n",hostname);
\end{lstlisting}
\\*
The array "char hostname[256]" is to store the input of the user
\\*
In "scanf" we need to limit the size of the input (256 bytes) to avoid buffer overflow attack.
\\*
\begin{lstlisting}
   host_name = gethostbyname(hostname);
    if (host_name == NULL){
        printf("Invalid host name!\n");
        exit(1);
    }
\end{lstlisting}
\\*

This step is to check the host name that user input is valid or not. If it not valid, it will send a message and close the program
\\*
\textbf{2.3 Resolved the Ip address (IPv4)}
\begin{lstlisting}
   else{
        for(int i=0; host_name->h_addr_list[i] != NULL; i++){
            printf("IP address: %s \n", inet_ntoa(*(struct in_addr *) (host_name->h_addr_list[i])));
        }
    }
\end{lstlisting}
This code is to print all the IP address of a domain name

\section{Result}
\textbf{3.1 On laptop}
\\*
First, we complile the file:
\begin{verbatim}
    gcc 02.practical.work.gethostbyname.c
\end{verbatim}
Then run it with the input: "google.com"
\begin{verbatim}
    ./a.out
\end{verbatim}
And the result:
\begin{verbatim}
    Enter host domain name: google.com
    Host name : google.com 
    IP address: 172.217.25.14 
\end{verbatim}

\textbf{3.2 On VPS}
\\*
First, we need to connect to the server
\begin{verbatim}
    root@{server ip address}
\end{verbatim}
\\*
Then we compile the C file and run it like we did above
\\*
With the input :"google.com"
\begin{verbatim}
    Enter host domain name: google.com
    Host name : google.com 
    IP address: 142.251.10.139 
    IP address: 142.251.10.138 
    IP address: 142.251.10.113 
    IP address: 142.251.10.102 
    IP address: 142.251.10.100 
    IP address: 142.251.10.101
\end{verbatim}
\\*
\textbf{3.3 Explain}
\\*
As you can see, the result on my local laptop and on my VPS is different. Because the geological distance can make a domain have a multiple addresses.

\end{document}
\documentclass{article}
\usepackage[utf8]{inputenc}
\usepackage{listings}
\usepackage{color}


\definecolor{dkgreen}{rgb}{0,0.6,0}
\definecolor{gray}{rgb}{0.5,0.5,0.5}
\definecolor{mauve}{rgb}{0.58,0,0.82}

\lstset{frame=tb,
  language=C,
  aboveskip=3mm,
  belowskip=3mm,
  showstringspaces=false,
  columns=flexible,
  basicstyle={\small\ttfamily},
  numbers=none,
  numberstyle=\tiny\color{gray},
  keywordstyle=\color{blue},
  commentstyle=\color{dkgreen},
  stringstyle=\color{mauve},
  breaklines=true,
  breakatwhitespace=true,
  tabsize=3
}

\title{Practical 2}
\author{Nguyen Tien Cong}
\date{May 2022}


\begin{document}

\maketitle

\section{Introduction}
In this practical work, we write a C program that show the IP address of the domain name.
\\*
- Use the function gethostbyname() to resolve a domain name.
\\* - Show the resolved IP address

\section{Implementation}
\textbf{2.1 Include library}
\\*
This is very important step, without library, the program won't able to be executed!
\\*
Here is all the library that we need to include in out program: 
\begin{lstlisting}
#include <stdio.h>
#include <stdlib.h>
#include <unistd.h>
#include <errno.h>
#include <netdb.h>
#include <sys/types.h>
#include <sys/socket.h>
#include <netinet/in.h>
#include <arpa/inet.h>
#include <string.h>
\end{lstlisting}
\\*
\textbf{2.2 Get input name from user}
\begin{lstlisting}
   char hostname[256];
   scanf("%256s", hostname);
   printf("Host name : %s \n",hostname);
\end{lstlisting}
\\*
The array "char hostname[256]" is to store the input of the user
\\*
In "scanf" we need to limit the size of the input (256 bytes) to avoid buffer overflow attack.
\\*
\begin{lstlisting}
   host_name = gethostbyname(hostname);
    if (host_name == NULL){
        printf("Invalid host name!\n");
        exit(1);
    }
\end{lstlisting}
\\*

This step is to check the host name that user input is valid or not. If it not valid, it will send a message and close the program
\\*
\textbf{2.3 Resolved the Ip address (IPv4)}
\begin{lstlisting}
   else{
        for(int i=0; host_name->h_addr_list[i] != NULL; i++){
            printf("IP address: %s \n", inet_ntoa(*(struct in_addr *) (host_name->h_addr_list[i])));
        }
    }
\end{lstlisting}
This code is to print all the IP address of a domain name

\section{Result}
\textbf{3.1 On laptop}
\\*
First, we complile the file:
\begin{verbatim}
    gcc 02.practical.work.gethostbyname.c
\end{verbatim}
Then run it with the input: "google.com"
\begin{verbatim}
    ./a.out
\end{verbatim}
And the result:
\begin{verbatim}
    Enter host domain name: google.com
    Host name : google.com 
    IP address: 172.217.25.14 
\end{verbatim}

\textbf{3.2 On VPS}
\\*
First, we need to connect to the server
\begin{verbatim}
    root@{server ip address}
\end{verbatim}
\\*
Then we compile the C file and run it like we did above
\\*
With the input :"google.com"
\begin{verbatim}
    Enter host domain name: google.com
    Host name : google.com 
    IP address: 142.251.10.139 
    IP address: 142.251.10.138 
    IP address: 142.251.10.113 
    IP address: 142.251.10.102 
    IP address: 142.251.10.100 
    IP address: 142.251.10.101
\end{verbatim}
\\*
\textbf{3.3 Explain}
\\*
As you can see, the result on my local laptop and on my VPS is different. Because the geological distance can make a domain have a multiple addresses.

\end{document}
\documentclass{article}
\usepackage[utf8]{inputenc}
\usepackage{listings}
\usepackage{color}


\definecolor{dkgreen}{rgb}{0,0.6,0}
\definecolor{gray}{rgb}{0.5,0.5,0.5}
\definecolor{mauve}{rgb}{0.58,0,0.82}

\lstset{frame=tb,
  language=C,
  aboveskip=3mm,
  belowskip=3mm,
  showstringspaces=false,
  columns=flexible,
  basicstyle={\small\ttfamily},
  numbers=none,
  numberstyle=\tiny\color{gray},
  keywordstyle=\color{blue},
  commentstyle=\color{dkgreen},
  stringstyle=\color{mauve},
  breaklines=true,
  breakatwhitespace=true,
  tabsize=3
}

\title{Practical 2}
\author{Nguyen Tien Cong}
\date{May 2022}


\begin{document}

\maketitle

\section{Introduction}
In this practical work, we write a C program that show the IP address of the domain name.
\\*
- Use the function gethostbyname() to resolve a domain name.
\\* - Show the resolved IP address

\section{Implementation}
\textbf{2.1 Include library}
\\*
This is very important step, without library, the program won't able to be executed!
\\*
Here is all the library that we need to include in out program: 
\begin{lstlisting}
#include <stdio.h>
#include <stdlib.h>
#include <unistd.h>
#include <errno.h>
#include <netdb.h>
#include <sys/types.h>
#include <sys/socket.h>
#include <netinet/in.h>
#include <arpa/inet.h>
#include <string.h>
\end{lstlisting}
\\*
\textbf{2.2 Get input name from user}
\begin{lstlisting}
   char hostname[256];
   scanf("%256s", hostname);
   printf("Host name : %s \n",hostname);
\end{lstlisting}
\\*
The array "char hostname[256]" is to store the input of the user
\\*
In "scanf" we need to limit the size of the input (256 bytes) to avoid buffer overflow attack.
\\*
\begin{lstlisting}
   host_name = gethostbyname(hostname);
    if (host_name == NULL){
        printf("Invalid host name!\n");
        exit(1);
    }
\end{lstlisting}
\\*

This step is to check the host name that user input is valid or not. If it not valid, it will send a message and close the program
\\*
\textbf{2.3 Resolved the Ip address (IPv4)}
\begin{lstlisting}
   else{
        for(int i=0; host_name->h_addr_list[i] != NULL; i++){
            printf("IP address: %s \n", inet_ntoa(*(struct in_addr *) (host_name->h_addr_list[i])));
        }
    }
\end{lstlisting}
This code is to print all the IP address of a domain name

\section{Result}
\textbf{3.1 On laptop}
\\*
First, we complile the file:
\begin{verbatim}
    gcc 02.practical.work.gethostbyname.c
\end{verbatim}
Then run it with the input: "google.com"
\begin{verbatim}
    ./a.out
\end{verbatim}
And the result:
\begin{verbatim}
    Enter host domain name: google.com
    Host name : google.com 
    IP address: 172.217.25.14 
\end{verbatim}

\textbf{3.2 On VPS}
\\*
First, we need to connect to the server
\begin{verbatim}
    root@{server ip address}
\end{verbatim}
\\*
Then we compile the C file and run it like we did above
\\*
With the input :"google.com"
\begin{verbatim}
    Enter host domain name: google.com
    Host name : google.com 
    IP address: 142.251.10.139 
    IP address: 142.251.10.138 
    IP address: 142.251.10.113 
    IP address: 142.251.10.102 
    IP address: 142.251.10.100 
    IP address: 142.251.10.101
\end{verbatim}
\\*
\textbf{3.3 Explain}
\\*
As you can see, the result on my local laptop and on my VPS is different. Because the geological distance can make a domain have a multiple addresses.

\end{document}
\documentclass{article}
\usepackage[utf8]{inputenc}
\usepackage{listings}
\usepackage{color}


\definecolor{dkgreen}{rgb}{0,0.6,0}
\definecolor{gray}{rgb}{0.5,0.5,0.5}
\definecolor{mauve}{rgb}{0.58,0,0.82}

\lstset{frame=tb,
  language=C,
  aboveskip=3mm,
  belowskip=3mm,
  showstringspaces=false,
  columns=flexible,
  basicstyle={\small\ttfamily},
  numbers=none,
  numberstyle=\tiny\color{gray},
  keywordstyle=\color{blue},
  commentstyle=\color{dkgreen},
  stringstyle=\color{mauve},
  breaklines=true,
  breakatwhitespace=true,
  tabsize=3
}

\title{Practical 2}
\author{Nguyen Tien Cong}
\date{May 2022}


\begin{document}

\maketitle

\section{Introduction}
In this practical work, we write a C program that show the IP address of the domain name.
\\*
- Use the function gethostbyname() to resolve a domain name.
\\* - Show the resolved IP address

\section{Implementation}
\textbf{2.1 Include library}
\\*
This is very important step, without library, the program won't able to be executed!
\\*
Here is all the library that we need to include in out program: 
\begin{lstlisting}
#include <stdio.h>
#include <stdlib.h>
#include <unistd.h>
#include <errno.h>
#include <netdb.h>
#include <sys/types.h>
#include <sys/socket.h>
#include <netinet/in.h>
#include <arpa/inet.h>
#include <string.h>
\end{lstlisting}
\\*
\textbf{2.2 Get input name from user}
\begin{lstlisting}
   char hostname[256];
   scanf("%256s", hostname);
   printf("Host name : %s \n",hostname);
\end{lstlisting}
\\*
The array "char hostname[256]" is to store the input of the user
\\*
In "scanf" we need to limit the size of the input (256 bytes) to avoid buffer overflow attack.
\\*
\begin{lstlisting}
   host_name = gethostbyname(hostname);
    if (host_name == NULL){
        printf("Invalid host name!\n");
        exit(1);
    }
\end{lstlisting}
\\*

This step is to check the host name that user input is valid or not. If it not valid, it will send a message and close the program
\\*
\textbf{2.3 Resolved the Ip address (IPv4)}
\begin{lstlisting}
   else{
        for(int i=0; host_name->h_addr_list[i] != NULL; i++){
            printf("IP address: %s \n", inet_ntoa(*(struct in_addr *) (host_name->h_addr_list[i])));
        }
    }
\end{lstlisting}
This code is to print all the IP address of a domain name

\section{Result}
\textbf{3.1 On laptop}
\\*
First, we complile the file:
\begin{verbatim}
    gcc 02.practical.work.gethostbyname.c
\end{verbatim}
Then run it with the input: "google.com"
\begin{verbatim}
    ./a.out
\end{verbatim}
And the result:
\begin{verbatim}
    Enter host domain name: google.com
    Host name : google.com 
    IP address: 172.217.25.14 
\end{verbatim}

\textbf{3.2 On VPS}
\\*
First, we need to connect to the server
\begin{verbatim}
    root@{server ip address}
\end{verbatim}
\\*
Then we compile the C file and run it like we did above
\\*
With the input :"google.com"
\begin{verbatim}
    Enter host domain name: google.com
    Host name : google.com 
    IP address: 142.251.10.139 
    IP address: 142.251.10.138 
    IP address: 142.251.10.113 
    IP address: 142.251.10.102 
    IP address: 142.251.10.100 
    IP address: 142.251.10.101
\end{verbatim}
\\*
\textbf{3.3 Explain}
\\*
As you can see, the result on my local laptop and on my VPS is different. Because the geological distance can make a domain have a multiple addresses.

\end{document}

\begin{verbatim}
    Enter host domain name: google.com
    Host name : google.com 
    IP address: 142.251.10.139 
    IP address: 142.251.10.138 
    IP address: 142.251.10.113 
    IP address: 142.251.10.102 
    IP address: 142.251.10.100 
    IP address: 142.251.10.101
\end{verbatim}
\\*
\textbf{3.3 Explain}
\\*
As you can see, the result on my local laptop and on my VPS is different. Because google has different IP address in the world to enable fast routing to nearest server.

\end{document}